\documentclass[10pt]{article}
\usepackage{graphicx} % Required for inserting images
\usepackage{titlesec} % Required for formatting the title
\usepackage{titling} % Required for calling author in maketitle
\usepackage{hyperref} % Required for hyperlinks
\usepackage{geometry} % Required for resizing the layout of the page
% Required for formatting
\usepackage{enumitem}
\usepackage{parskip}  

\pagestyle{empty} % Removes the page numbers
% Title Formatting
\titleformat{\section}{\Large\bfseries\vspace{-0.4em}}{}{0em}{} [\titlerule]
\titleformat{\subsection}{\large\bfseries\vspace{-0.4em}}{}{0em}{}
\author{Nicholas Caro Lopez}
\date{May 2023}

% Creates a command for right-indenting text
\newcommand{\rightindent}[1]{\hfill{\textmd{#1}}}
% Creates a command for handling jobs
\newcommand{\job}[3]{
    #1,  % 1st Arg = Role Title
    \textmd{#2} % 2nd Arg = Company
    \hfill{\textmd{#3}} % 3rd arg = Duration
}

% Hyperlink Formatting
\hypersetup{
    colorlinks=true,
    urlcolor=blue
}
% geometry layout setup
\geometry{
    letterpaper,
    left=.5in,
    top=.7in,
    right=.5in,
    bottom=.7in
}

% Redefines the maketitle command
\renewcommand{\maketitle}{
    \begin{center}
    {\LARGE\bfseries\theauthor\\} % Generates name, in bold and LARGE text size and a newline char
    \vspace{0.3em} % Adds some whitespace
    Burlington, ON $\mid$ \href{https://nickcarolopez.ca/}{Website}  $\mid$ \href{https://linkedin.com/in/nicholas-caro-lopez-123738252/}{LinkedIn} $\mid$ \href{https://github.com/nicholas489}{Github}
    \end{center}
    \vspace{-1.2em} % Removes whitespace
}

% Removes the spacing between each bullet point in itemize & rest of document
\let\olditemize\itemize
\let\oldenditemize\enditemize
\renewcommand{\itemize}{
    \vspace{-0.4em}
    \olditemize
    \setlength{\parskip}{0em}
}
\renewcommand{\enditemize}{
    \oldenditemize
    \vspace{-0.7em}
}

\begin{document}
\maketitle

\section{Education}
\subsection{Toronto Metropolitan University (formerly Ryerson University) \rightindent{Sept. 2022 - Present}}
\vspace{-0.3em}Candidate for Bachelor of Science (Honours), Computer Science Co-op \rightindent{CGPA: 4.04/4.33}
\begin{itemize}
    \item \textbf{Relevant Coursework}: Introduction to Software Engineering, Data Structures, Operating Systems I, Web Systems Development, Intro to Unix \& C \& C++, Probability and Statistics, Linear Algebra
    \item \textbf{Extracurricular Involvement}: Senior Representative for Metropolitan Open Source Society
    \item \textbf{Awards}: Faculty of Science Dean's List 2022-23, Academic Entrance Scholarship 
\end{itemize}

\section{Technical Skills}
\begin{itemize}
    \item \textbf{Computer Languages}: Python, Java, C, HTML, CSS, JavaScript, Lisp, Perl, PHP, Ruby
    \item \textbf{Frameworks/Libraries}: MySQL, Tailwind CSS, Bootstrap, JQuery, PyQt
    \item \textbf{Development Tools}: Visual Studio Code, Git/Github, Unix/Linux
    \item \textbf{Spoken Languages}: English, Spanish
\end{itemize}

\section{Work Experience}
\subsection{\job{Software Developer}{Toronto Metropolitan University - Toronto, ON}{Sept. 2023 - Present}}
\begin{itemize}
    \item Integrated an advanced image-based motion tracking extension into \textbf{open-source software}, 3D Slicer
    \item Cooperated effectively within a \textbf{Agile software development team}, optimizing workflow, visualizing progress, and delivering quality software solutions on time
    \item Reworked the \textbf{Backend Logic} of the extension, enabling multi-file deletion and processing within a single session
    \item Created new functionality on the graphical user interface, increasing usability by 28\%, using the \textbf{3D Slicer Python API}
\end{itemize}
\subsection{\job{Teaching Assistant}{RoboThink Northwest Toronto - Burlington, ON}{May 2023 - Aug. 2023}}
\begin{itemize}
    \item Designed and taught a comprehensive curriculum introducing students to programming in Scratch 
    \item Assisted the STEM Teacher in utilizing RoboThink materials effectively to deliver engaging and interactive lessons
    \item Enhanced students learning through administrative tasks such as organizing curriculum binders and printouts, improving productivity by 50\%
    \item Demonstrated attentiveness to individual students, providing personalized support and guidance to enhance their learning experience
\end{itemize}
\subsection{\job{Associate}{Dollarama - Burlington, ON}{May 2022 - Aug. 2022}}
\begin{itemize} 
    \item Provided courteous customer service by warmly welcoming customers and addressing their inquiries or issues promptly
    \item Conducted financial transactions including credit/debit cards, cash, gift cards, and store credits with proficiency
\end{itemize}

\section{Personal Projects}
\subsection{\href{https://nicholas489.github.io/Sudoku_Solver/}{Sudoku Solver} \textmd{- HTML, CSS, JavaScript} \rightindent{May 2023 - Aug. 2023}}
\begin{itemize}
    \item Defined an \textbf{Event Listener} and \textbf{Regex} for capturing and restricting user input to digits 1-9
    \item Created a \textbf{responsive} and \textbf{interactive web page} using CSS concepts, namely \textbf{Media Queries} and \textbf{Flexbox/Grid Layout}
    \item Employed the \textbf{Backtracking Algorithm} to solve every valid Sudoku board inputted by the user and displayed the solution dynamically on the web page
\end{itemize}
\subsection{Simon-Says Memory Game \textmd{- Java} \rightindent{Feb. 2023 - Apr. 2023}}
\begin{itemize} 
    \item Utilized the \textbf{JFrame} library to generate and display the graphical user interface
    \item Incorporated \textbf{Object-Oriented Programming} concepts such as \textbf{Classes}, \textbf{Objects} and \textbf{Inheritance} to create and display items such as the game level and buttons
    \item Applied \textbf{Inheritance} and \textbf{Method Overriding} enabling the program to receive the user’s mouse clicks and execute certain functions accordingly
\end{itemize}
\subsection{Riemann Sum Calculator \textmd{- Python} \rightindent{Oct. 2022 - Dec. 2022}}
\begin{itemize} 
    \item Wrote a script with the purpose of approximating the area under a polynomial curve
    \item Implemented \textbf{Variables}, \textbf{Data Types}, \textbf{Control Flow}, \textbf{Modules} and \textbf{User Input} to read a given mathematical function
    \item Executed \textbf{Functions}, \textbf{Exception Handling} and \textbf{File I/O} to calculate the integral and return the area in a .txt file
\end{itemize}
\end{document}